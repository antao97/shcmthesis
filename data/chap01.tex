% !TeX root = ../shcmthesis-example.tex

\part{\LaTeX{} 使用}

\chapter{\LaTeX{} 介绍}

\textbf{声明:以下文字节选复制自\url{https://oi-wiki.org/tools/latex/},并根据本文情况稍作修改}

\section{什么是 \LaTeX{}}

LaTeX(读作/ˈlɑːtɛx/或/ˈleɪtɛx/)是一个让你的文档看起来更专业的排版系统,而不是文字处理器。它尤其适合处理篇幅较长、结构严谨的文档,并且十分擅长处理公式表达。它是免费的软件,对大多数操作系统都适用。

LaTeX 基于 TeX(Donald Knuth 在 1978 年为数字化排版设计的排版系统)。TeX 是一种电脑能够处理的低级语言,但大多数人发现它很难使用。LaTeX 正是为了让它变得更加易用而设计的。目前 LaTeX 的版本是 LaTeX 2e。

如果你习惯于使用微软的 Office Word 处理文档,那么你会觉得 LaTeX 的工作方式让你很不习惯。Word 是典型的「所见即所得」的编辑器,你可以在编排文档的时侯查看到最终的排版效果。但使用 LaTeX 时你并不能方便地查看最终效果,这使得你专注于内容而不是外观的调整。

一个 LaTeX 文档是一个以 .tex 结尾的文本文件,可以使用任意的文本编辑器编辑,比如 Notepad,但对于大多数人而言,使用一个合适的 LaTeX 编辑器会使得编辑的过程容易很多。在编辑的过程中你可以标记文档的结构。完成后你可以进行编译——这意味着将它转化为另一种格式的文档。它支持多种格式,但最常用的是 PDF 文档格式。

\section{章节}

文档可以分为篇(Part)、章(Chatpers)、节(Sections)小节(Subsections)和小小节(Subsubsections),使用命令如下:

\begin{python}
\part{...}
\chapter{...}
\section{...}
\subsection{...}
\subsubsection{...}
\end{python}

\section{文字处理}

LaTeX 有多种不同的字体效果,在此列举一部分:

\begin{python}
\textit{words in italics} 
\textsl{words slanted} 
\textsc{words in smallcaps} 
\textbf{words in bold} 
\texttt{words in teletype} 
\textsf{sans serif words} 
\textrm{romanwords} 
\underline{underlined words}
\end{python}

效果如下:

\textit{words in italics} 

\textsl{words slanted} 

\textsc{words in smallcaps} 

\textbf{words in bold} 

\texttt{words in teletype} 

\textsf{sans serif words} 

\textrm{romanwords} 

\underline{underlined words}

其中,这些命令都可以通过快捷键打出,选中一句文字后按下快捷键即可,例如\textbackslash textit\{\} 为Ctrl+i,\textbackslash textbf\{\} 为Ctrl+b,
如果使用苹果电脑则将Ctrl替换为Command。

\section{段落}

\subsection{段落换行}
在LaTex中段落需要间隔一行以进行换行,例如:

\begin{python}
段落一
段落二

段落三
\end{python}

显示效果为:

段落一
段落二

段落三

\subsection{段落缩进}

LaTeX 默认每个章节第一段首行顶格,之后的段落首行缩进。如果想要段落顶格,在要顶格的段落前加 \noindent 命令即可。

\section{列表}

LaTeX 支持两种类型的列表:有序列表(enumerate)和无序列表(itemize)。列表中的元素定义为\textbackslash item。列表可以有子列表。

\rightarrow 输入下面的内容来生成一个有序列表套无序列表:

\begin{python}
\begin{enumerate}
	\item First thing
	
	\item Second thing
	\begin{itemize}
		\item A sub-thing
		
		\item Another sub-thing
	\end{itemize}
	
	\item Third thing
\end{enumerate}
\end{python}

\rightarrow 编译并核对 PDF 文档。

列表长这样:

\begin{enumerate}
	\item First thing
	
	\item Second thing
	\begin{itemize}
		\item A sub-thing
		
		\item Another sub-thing
	\end{itemize}
	
	\item Third thing
\end{enumerate}

可以使用方括号参数来修改无序列表头的标志。例如,\textbackslash item[-] 会使用一个杠作为标志,你甚至可以使用一个单词,比如 \textbackslash item[One]。

下面的代码:

\begin{python}
\begin{itemize}
	\item[-] First thing
	
	\item[+] Second thing
	\begin{itemize}
		\item[Fish] A sub-thing
		
		\item[Plants] Another sub-thing
	\end{itemize}
	
	\item[Q] Third thing
\end{itemize}
\end{python}


生成的效果为:

\begin{itemize}
	\item[-] First thing
	
	\item[+] Second thing
	\begin{itemize}
		\item[Fish] A sub-thing
		
		\item[Plants] Another sub-thing
	\end{itemize}
	
	\item[Q] Third thing
\end{itemize}

\section{注释和空格}

我们使用 \% 创建一个单行注释,在这个字符之后的该行上的内容都会被忽略,直到下一行开始。快捷键为Ctrl+/,若使用苹果电脑则为Command+/。

下面的代码:

\begin{python}
It is a truth universally acknowledged % Note comic irony
in the very first sentence , 
that a single man in possession of a good fortune,
must be in want of a wife.
\end{python}

生成的结果为:

It is a truth universally acknowledged % Note comic irony
in the very first sentence , that a single man in possession of a good fortune,
must be in want of a wife.

多个连续空格在 LaTeX 中被视为一个空格。多个连续空行被视为一个空行。空行的主要功能是开始一个新的段落。通常来说,LaTeX 忽略空行和其他空白字符,两个反斜杠(\textbackslash\textbackslash)可以被用来换行。

如果你想要在你的文档中添加空格,你可以通过添加~从而添加空格。
或者 \textbackslash vaspace\{...\} 的命令,这样可以添加竖着的空格,高度可以指定。如 \textbackslash vspace\{12pt\} 会产生一个空格,高度等于 12pt 的文字的高度。

\section{脚注}

在需要添加脚注的地方插入\textbackslash footnote{脚注内容},则会自动在页脚部分插入脚注内容。

例如:

\begin{python}
文字内容一\footnote{注释一}和文字内容二\footnote{注释二}
\end{python}

效果如下:

文字内容一\footnote{注释一}和文字内容二\footnote{注释二}

可以看到在本页底部已经自动按编号添加了注释内容。

\section{特殊字符}

下列字符在 LaTeX 中属于特殊字符:

\begin{python}
#
$
%
^
&
_
{
}
~
\
\end{python}

为了使用这些字符,我们需要在他们前面添加反斜杠进行转义:

\begin{python}
\#
\$
\%
\^{}
\&
\_
\{
\}
\~{}
\end{python}

\section{表格}

表格(tabular)命令用于排版表格,表应具有自明性。

\subsection{普通表格}

LaTeX 默认表格是没有横向和竖向的分割线的——如果你需要,你得手动设定。LaTeX 会根据内容自动设置表格的宽度。

下面的代码可以创一个表格:

\begin{python}
\begin{tabular}{...}
\end{tabular}
\end{python}
	
例如:

\begin{python}
\begin{center}
\begin{tabular}{|r|l|}
	\centering
	\hline
	8              & here's \\
	\cline 86 & stuff  \\
	\hline
	\hline
	2008           & now    \\
	\hline
\end{tabular}
\end{center}
\end{python}

效果如下:

\begin{center}
\begin{tabular}{|r|l|}
	\hline
	8              & here's \\
	\cline{2-2}
	86 & stuff  \\
	\hline
	2008           & now    \\
	\hline
\end{tabular}
\end{center}

l 表示一个左对齐的列;

r 表示一个右对齐的列;

c 表示一个向中对齐的列;

| 表示一个列的竖线;

例如,\{lll\} 会生成一个三列的表格,并且保存向左对齐,没有显式的竖线;\{|l|l|r|\} 会生成一个三列表格,前两列左对齐,最后一列右对齐,并且相邻两列之间有显式的竖线。

表格的数据在 \textbackslash begin\{tabular\} 后输入:

\& 用于分割列;

\textbackslash \textbackslash 用于换行;

\textbackslash hline 表示插入一个贯穿所有列的横着的分割线;

\textbackslash cline\{1-2\} 会在第一列和第二列插入一个横着的分割线。

最后使用 \textbackslash end\{tabular\} 结束表格。

\subsection{三线表}

为使表格简洁易读,可以采用三线表,如表~\ref{tab:three-line}。
三条线可以使用 \pkg{booktabs} 宏包提供的命令生成。

例如:

\begin{python}
\begin{table}
	\centering
	\caption{三线表示例}
	\begin{tabular}{ll}
		\toprule
		文件名          & 描述                         \\
		\midrule
		shcmthesis.dtx   & 模板的源文件,包括文档和注释 \\
		shcmthesis.cls   & 模板文件                     \\
		shcmthesis-*.bst & BibTeX 参考文献表样式文件    \\
		\bottomrule
	\end{tabular}
	\label{tab:three-line}
\end{table}
\end{python}

效果如下:

\begin{table}
	\centering
	\caption{三线表示例}
	\begin{tabular}{ll}
		\toprule
		文件名          & 描述                         \\
		\midrule
		shcmthesis.dtx   & 模板的源文件,包括文档和注释 \\
		shcmthesis.cls   & 模板文件                     \\
		shcmthesis-*.bst & BibTeX 参考文献表样式文件    \\
		\bottomrule
	\end{tabular}
	\label{tab:three-line}
\end{table}

表格如果有附注,尤其是需要在表格中进行标注时,可以使用 \pkg{threeparttable} 宏包。
研究生要求使用英文小写字母 a、b、c……顺序编号,本科生使用圈码 ①、②、③……编号。

例如:

\begin{python}
\begin{table}
	\centering
	\begin{threeparttable}[c]
		\caption{带附注的表格示例}
		\label{tab:three-part-table}
		\begin{tabular}{ll}
			\toprule
			文件名                 & 描述                         \\
			\midrule
			shcmthesis.dtx\tnote{a} & 模板的源文件,包括文档和注释 \\
			shcmthesis.cls\tnote{b} & 模板文件                     \\
			shcmthesis-*.bst        & BibTeX 参考文献表样式文件    \\
			\bottomrule
		\end{tabular}
		\begin{tablenotes}
			\item [a] 可以通过 xelatex 编译生成模板的使用说明文档;
			使用 xetex 编译 \file{shcmthesis.ins} 时则会从 \file{.dtx} 中去除掉文档和注释,得到精简的 \file{.cls} 文件。
			\item [b] 更新模板时,一定要记得编译生成 \file{.cls} 文件,否则编译论文时载入的依然是旧版的模板。
		\end{tablenotes}
	\end{threeparttable}
\end{table}
\end{python}

效果如下:

\begin{table}
	\centering
	\begin{threeparttable}[c]
		\caption{带附注的表格示例}
		\label{tab:three-part-table}
		\begin{tabular}{ll}
			\toprule
			文件名                 & 描述                         \\
			\midrule
			shcmthesis.dtx\tnote{a} & 模板的源文件,包括文档和注释 \\
			shcmthesis.cls\tnote{b} & 模板文件                     \\
			shcmthesis-*.bst        & BibTeX 参考文献表样式文件    \\
			\bottomrule
		\end{tabular}
		\begin{tablenotes}
			\item [a] 可以通过 xelatex 编译生成模板的使用说明文档;
			使用 xetex 编译 \file{shcmthesis.ins} 时则会从 \file{.dtx} 中去除掉文档和注释,得到精简的 \file{.cls} 文件。
			\item [b] 更新模板时,一定要记得编译生成 \file{.cls} 文件,否则编译论文时载入的依然是旧版的模板。
		\end{tablenotes}
	\end{threeparttable}
\end{table}

如某个表需要转页接排,可以使用 \pkg{longtable} 宏包,需要在随后的各页上重复表的编号。
编号后跟表题(可省略)和“(续)”,置于表上方。续表均应重复表头。

例如:

\begin{python}
\begin{longtable}{cccc}
	\caption{跨页长表格的表题}
	\label{tab:longtable} \\
	\toprule
	表头 1 & 表头 2 & 表头 3 & 表头 4 \\
	\midrule
	\endfirsthead
	\caption*{续表~\thetable\quad 跨页长表格的表题} \\
	\toprule
	表头 1 & 表头 2 & 表头 3 & 表头 4 \\
	\midrule
	\endhead
	\bottomrule
	\endfoot
	Row 1  & & & \\
	Row 2  & & & \\
	Row 3  & & & \\
	Row 4  & & & \\
	Row 5  & & & \\
	Row 6  & & & \\
	Row 7  & & & \\
	Row 8  & & & \\
	Row 9  & & & \\
	Row 10 & & & \\
\end{longtable}
\end{python}

效果如下:

\begin{longtable}{cccc}
	\caption{跨页长表格的表题}
	\label{tab:longtable} \\
	\toprule
	表头 1 & 表头 2 & 表头 3 & 表头 4 \\
	\midrule
	\endfirsthead
	\caption*{续表~\thetable\quad 跨页长表格的表题} \\
	\toprule
	表头 1 & 表头 2 & 表头 3 & 表头 4 \\
	\midrule
	\endhead
	\bottomrule
	\endfoot
	Row 1  & & & \\
	Row 2  & & & \\
	Row 3  & & & \\
	Row 4  & & & \\
	Row 5  & & & \\
	Row 6  & & & \\
	Row 7  & & & \\
	Row 8  & & & \\
	Row 9  & & & \\
	Row 10 & & & \\
\end{longtable}

\subsection{表格引用}

通过在表格中加入\textbackslash label\{tab\_name\}来定义表格名称,在文本中使用\textbackslash ref\{tab\_name\}来引用。

建议在\textbackslash ref前添加空格~以示美观。

例如:

\begin{python}
此处引用表~\ref{tab:longtable}。
\end{python}

效果为:

此处引用表~\ref{tab:longtable}。

\section{图片}

本章介绍如何在 LaTeX 文档中插入图片。

图片应当是 PDF,PNG,JPEG 或者 GIF 文件。
建议矢量图片使用 PDF 格式,比如数据可视化的绘图;
照片应使用 JPG 格式;
其他的栅格图应使用无损的 PNG 格式。
注意,LaTeX 不支持 TIFF 格式;EPS 格式已经过时。

\subsection{单张图片}

图片通常在 \env{figure} 环境中使用 \cs{includegraphics} 插入,如图~\ref{fig:example} 的源代码。

例如:

\begin{python}
\begin{figure}
	\centering
	\includegraphics[width=0.5\linewidth]{example-image-a.pdf}
	\caption*{国外的期刊习惯将图表的标题和说明文字写成一段,需要改写为标题只含图表的名称,其他说明文字以注释方式写在图表下方,或者写在正文中。}
	\caption{示例图片标题}
	\label{fig:example}
\end{figure}
\end{python}

效果如下:

\begin{figure}
	\centering
	\includegraphics[width=0.5\linewidth]{example-image-a.pdf}
	\caption*{国外的期刊习惯将图表的标题和说明文字写成一段,需要改写为标题只含图表的名称,其他说明文字以注释方式写在图表下方,或者写在正文中。}
	\caption{示例图片标题}
	\label{fig:example}
\end{figure}

[h] 是位置参数,h 表示把图表近似地放置在这里(如果能放得下)。有其他的选项:t 表示放在在页面顶端;b 表示放在在页面的底端;p 表示另起一页放置图表。你也可以添加一个 ! 参数来强制放在参数指定的位置(尽管这样排版的效果可能不太好)。

\textbackslash centering 将图片放置在页面的中央。如果没有该命令会默认左对齐。使用它的效果是很好的,因为图表的标题也是居中对齐的。

\textbackslash includegraphics\{...\} 命令可以自动将图放置到你的文档中,图片文件应当与 TeX 文件放在同一目录下。

width=1\textbackslash textwidth 是一个可选的参数,它指定图片的宽度——与文本的宽度相同。宽度也可以以厘米为单位。你也可以使用 [scale=0.5] 将图片按比例缩小(示例相当于缩小一半)。

\textbackslash caption\{...\} 定义了图表的标题。如果使用了它,LaTeX 会给你的图表添加「Figure」开头的序号。你可以使用 \textbackslash listoffigures 来生成一个图表的目录。

\textbackslash label\{...\} 创建了一个可以供你引用的标签。

若图或表中有附注,采用英文小写字母顺序编号,附注写在图或表的下方。
国外的期刊习惯将图表的标题和说明文字写成一段,需要改写为标题只含图表的名称,其他说明文字以注释方式写在图表下方,或者写在正文中。

\subsection{多个图片}

如果一个图由两个或两个以上分图组成时,各分图分别以 (a)、(b)、(c)...... 作为图序,并须有分图题。
推荐使用 \pkg{subcaption} 宏包来处理, 比如图~\ref{fig:multi-image}中的图~\ref{fig:subfig-a} 和图~\ref{fig:subfig-b}和。

例如:

\begin{python}
\begin{figure}
	\centering
	\subcaptionbox{分图 A\label{fig:subfig-a}}
	{\includegraphics[width=0.35\linewidth]{example-image-a.pdf}}
	\subcaptionbox{分图 B\label{fig:subfig-b}}
	{\includegraphics[width=0.35\linewidth]{example-image-b.pdf}}
	\caption{多个分图的示例}
	\label{fig:multi-image}
\end{figure}
\end{python}

效果如下:

\begin{figure}
	\centering
	\subcaptionbox{分图 A\label{fig:subfig-a}}
	{\includegraphics[width=0.35\linewidth]{example-image-a.pdf}}
	\subcaptionbox{分图 B\label{fig:subfig-b}}
	{\includegraphics[width=0.35\linewidth]{example-image-b.pdf}}
	\caption{多个分图的示例}
	\label{fig:multi-image}
\end{figure}

\subsection{图片引用}

通过在表格中加入\textbackslash label\{fig\_name\}来定义表格名称,在文本中使用\textbackslash ref\{fig\_name\}来引用。

建议在\textbackslash ref前添加空格~以示美观。

例如:

\begin{python}
此处引用图~\ref{fig:multi-image},及其图~\ref{fig:subfig-b}和图~\ref{fig:subfig-b}。
\end{python}

效果为:

此处引用图~\ref{fig:multi-image},及其图~\ref{fig:subfig-b}和图~\ref{fig:subfig-b}。

\section{谱例}

对于音乐类论文,常需要使用谱例。本模板提供了专门的 \env{score} 环境来处理谱例,用法与 \env{figure} 环境类似,但编号和标题使用「谱例」前缀。

例如:

\begin{python}
\begin{score}
	\centering
	\includegraphics[width=0.8\linewidth]{example-image-a.pdf}
	\caption{示例谱例标题}
	\label{score:example}
\end{score}
\end{python}

效果如下:

\begin{score}
	\centering
	\includegraphics[width=0.8\linewidth]{example-image-a.pdf}
	\caption{示例谱例标题}
	\label{score:example}
\end{score}

谱例的引用方式与图片相同,使用 \cs{ref\{score:example\}} 即可引用谱例~\ref{score:example}。

谱例清单可以使用 \cs{listofscores} 命令生成,用法与 \cs{listoffigures} 类似。