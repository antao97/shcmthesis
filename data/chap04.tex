% !TeX root = ../shcmthesis-example.tex

\chapter{研究生学位论文基本要求}

\section{学位论文内容方面撰写要求}

学位论文应在导师指导下独立完成,且不能以在读期间该生导师及其作品作为学位论文研究对象,不得抄袭他人的文字或剽窃他人的研究成果。论文的选题和所研究的内容应在学术上有一定的理论意义和实践意义。学位申请人对论文所涉及的研究课题,应具有较为坚实的基础理论和专业知识。对前人在相同或相似课题上已经取得的研究成果应有比较全面的了解和认识。申请人应掌握论文研究课题所必须的研究方法和技能。

博士论文要求对所研究的课题具有系统性、学术性和前瞻性,在凸显本专业学科特性的同时,鼓励在研究方法上有所创新;学术型硕士论文要求对所研究的课题有新的见解,专业学位硕士论文要有一定的实践总结性。

论文的论据要充分翔实,所用材料须正确可靠。词句要精练,条理清晰;陈述和论证要有逻辑性;谱例图表运用要准确、恰当。论文中所有引用他人的词句,必须作相应的注释。引用文字与作者本人文字之间应保持合理的平衡,避免过度引用。若以综述或转述的方式引用他人的观点或研究成果,也须以注释的方式引证原文或提供原著作名称及相应的页码。对引文的注释内容至少应包括作者姓名、著作名称、出版社或期刊名称、出版或发表的日期和页码。凡大量引用他人词句而不加引号和注释的,均可视为抄袭行为。参考书目应列出作者姓名、著作名称、出版社名称、出版时间。中国古代文献要注明版本来源和出处。申请人在读期间所编著的教学参考书、校注、读书报告和对自己作品的分析研究等,均不得作为学位论文提交。

\section{学位论文摘要撰写要求}

研究生学位论文摘要是学位论文内容不加注释和评论的短篇陈述,应具有相对的独立性和完整性。研究生学位论文摘要应突出本文的新见解和创造性成果。

研究生学位论文摘要应包括中、英文两部分,英文摘要应是中文摘要的全文翻译。学位论文摘要无须单独装订,须装订在版权页(独创性声明)后。