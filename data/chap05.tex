% !TeX root = ../shcmthesis-example.tex

\chapter{研究生学位论文形式和规范要求}

\section{论文基本结构(按先后顺序排列)}

研究生学位论文应由封面、版权页(独创性声明和论文使用授权说明)、中英文摘要、关键词、目录、绪论、正文、结论、参考文献、附录(非必须)、致谢(或后记,非必须)、封底等部分依次组成。

\subsection{封页、纸张及页面设计}

论文封面和封底使用研究生部统一规定样式,可由本院研究生部网页下载版式,按提示的部位、字体、字号等格式要求进行填写(注:博士论文的英文为Dissertation,硕士论文的英文为Thesis)。其中,论文题目要简明扼要,引人注目,以不超过20字为宜,必要时可加用副标题。题名应该避免使用不常见的缩略词、首字母缩写字、字符、代号或公式等。

艺术硕士专业学位论文题目,正标题一般不超过20个字(可另加副标题)。论文题目中不要出现“浅谈”“初探”“浅论”等字样,可用“论”“研究”“探析”等表述。建议直接表述研究的对象和问题。

页面纸张尺寸选择A4型,即21厘米×29.7厘米。页面边距选择“普通”,即上下2.54厘米,左右3.18厘米。

\subsection{版权页(学位论文学术规范声明和学位论文使用授权书)}

为了维护学校、导师及研究生的合法权益,我院所有提出答辩申请的研究生都必须在其学位论文主体内容前附《上海音乐学院博(硕)学位论文学术规范声明》及《上海音乐学院博(硕)学位论文使用授权书》两项内容,以上两项内容放在一个版面页上。在答辩前提交学位论文和答辩后提交留存论文时,以上两项内容上应由研究生本人和导师在相应位置上签名。

\subsection{摘要}

摘要是论文内容的高度概括与简短陈述,应说明本文的目的、研究方法、成果和结论,要突出论文的创造性成果或新见解。语言力求精炼、准确。摘要文本应该避免引经据典与详加论述研究过程,以及添加表格、谱例、脚注等做法。中、英文摘要各占一页。

中文摘要:题头“摘要”二字位于页面上方居中,字间空两个字符,用黑体小二号字。段前、段后距各24磅或2行。博士、学术型硕士学位论文摘要篇幅以不超过1000字为限,用宋体五号字,两端对齐,行距为固定值20磅,段落首行缩进两个字符。艺术硕士专业学位摘要字数掌握在150-300字左右,用宋体五号字,两端对齐,行距为固定值20磅,段落首行缩进两个字符。

英文摘要:题头“Abstract”位于页面上方居中,用相当于中文小二号字加粗的Times New Roman字体。段前、段后距各24磅或2行。英文摘要内容应与中文摘要相对应,用相当于中文五号字的 Times New Roman 字体,两端对齐,行距为固定值20磅,段落首行缩进两个字3符。

英文摘要,包括正文中所出现英文或其他外语语种的段落,均须按照其语言习惯使用标点格式,不能按中文标点习惯使用。但是,如果在中文行文中出现的英语或其他语言的词语,其前后之间的标点,应按照中文标点习惯使用。

\subsection{关键词}

关键词是反映论文中最主要内容的术语,也是具有学术代表性的词汇,对文献检索有重要作用,一般是文中与主题紧密相关的术语,且应具备专业性、学术性、典型性、独特性的基本要求。关键词以3-6个为宜,分别列于中、英文摘要之后,与摘要之间间隔一行。英文关键词应与中文关键词相对应。

中文关键词之前的“关键词”后面用冒号“:”,中文关键词用宋体五号字,两个以上的关键词之间用分号分隔。

英文关键词之前的“Key Words”后面用冒号 “:”,英文关键词及标点符号用相当于中文五号字的Times New Roman字体,第一个字母大写,两个以上的关键词之间用分号分隔。

示例:

关键词:音乐学;方法论;体裁;赋格;主题

Key Words: Musicology; Methodology; Genre; Fugue; Theme

\subsection{目录}

目录应列出论文中篇(若分篇者)、章、节、条、附录等的序号、名称及页码。

题头“目录”二字位于页面上方居中,字间空两个字符,用黑体小二号字。

目条名中文数字序号“一”“二”为最高级,再下一级为“(一)(二)(三)”等,再次之为阿拉伯数字序号“1.2.3.”,又次之为“(1)(2)(3)”。作为论文目录,一般最多只列到第三级,即阿拉伯数字“1.2.3.”即可。

凡以文字表示章节的,序号后面不加标点,但章节序号与其后章节内文字间空两个半角字符,如“第一章 ××××”“第一节 ××××”。凡以数字表示顺序的,中文数字序号后面加顿号“、”,如“一、”;阿拉伯数字序号后面用小圆点“.”,如“1.”;序号用括弧者,其后不必加标点,如“(一)”“(1)”。

目录章节所示页码一律使用阿拉伯数字,所示章节名称用“……”链接,两端对齐。

若设有分篇,则目录中最多呈现共四级标题;若无分篇,则目录中最多呈现共三级标题,章节三级依次左缩进两个字符。

示例:

\noindent 第一章(顶格)

第一节(左缩进两个字符)

\indent \indent 一、(再左缩进两格) 或者

\indent \indent 1. (如上,再左缩进两个字符)

\subsection{绪论(或引言、或前言)}

综述本研究领域的国内外现状,提出本文所要解决的问题,新的见解和基本论点,以及本项研究工作的理论意义与实践价值等。

题头“绪论”(或“引言”或“前言”)位于页面上方居中,字间空两个半角字符,用黑体小二号字。引言文字用宋体五号字,以20磅行距与题头空一行。

一篇学位论文的引言大致包含如下几个部分:

1、问题的提出;

2、选题背 景及意义;

3、文献综述;

4、研究方法;

5、论文结构安排。

\begin{itemize}
	\item 问题的提出:要清晰地阐述所要研究的问题“是什么”。
	\footnote{选题时切记要有“问题意识”,不要选不是问题的问题来研究。}
	\item 选题背景及意义:论述清楚为什么选择这个题目来研究,即阐述该研究对学科发展的贡献、对国计民生的理论与现实意义等。
	\item 文献综述:对本研究主题范围内的文献进行详尽的综合述评,“述”的同时一定要有“评”,指出现有研究状态,仍存在哪些尚待解决的问题,讲出自己的研究有哪些探索性内容。
	\item 研究方法:讲清论文所使用的学术研究方法。
	\item 论文结构安排:介绍本论文的写作结构安排。
\end{itemize}

\subsection{正文}

正文为学位论文的主体。整体上两端对齐,行距为固定值20磅。

\subsubsection{正文标题}

篇名:若分篇者,篇名单独成页,上下左右居中,用黑体二号字。

章名:居中,用黑体小二号字,段前段后间距各24磅或2行。如有副标题,用仿宋、四号字、居中,段前间距0,段后间距24磅或2行。凡章名,应起于一页之首,一章结束,设分页符,下一章另起一页。

节名:居中,用楷体小三号字。段前、后间距24磅或2行。硕士学位论文若不设节名,则章名下直接用条名。

条名:左缩进两个字符,用宋体四号字,加粗,段前间距12磅或1行,段后为正文行距。条名使用标题序号应为“一、”“二、”“三、”等。

若条名下又有小标题,则格式同正文。条名以下各级标题序号,大小逻辑以及书写方式(尤其是标点)如下:

一、条名

\indent\indent (一)

\indent\indent 1.

\indent\indent (1)

\subsubsection{正文文字}

正文用宋体五号字,两端对齐,行距为固定值20磅,段落首行缩进两个字符。

\subsubsection{正文中引文}

凡论文中直接引用他人论著(或其他研究成果)的原文,必须在引文两端标注双引号“”,并在脚注即页下注中标明其出处\footnote{书名、作者、出版社},并在参考文献中列出书名、作者、出版社等;间接引用可不加引号,但也必须加脚注即页下注注明其出处。注释格式规范详见3.1。

若引文篇幅较长,可设为独立段。

未单独成段的引文,格式同正文。凡独立成段的引文,用\textit{楷体五号字},引文两端不需加引号。整段引文左右皆缩进两个字符;在引文结束处加脚注\footnote{出处}即页下注中标明其出处。

正文中的引文用\textit{楷体五号字},首行左缩进两个字符。

\subsection{结论}

要求明确、精练地总括本文的主要观点及新见解。

题头“结论”二字位于页面上方居中,字间空两个半字符,用黑体小二号字。结论文字同正文,亦用宋体五号字。

\subsection{参考文献}

参考文献指的是作者写作论文时所参考和引用的文献资料。

参考文献格式规范详见参考文献部分。

\subsection{附录}

对论文主旨的阐述有辅助或拓展作用的相关材料,但不便于放入正文的内容,如较大的表格、大篇幅谱例、图片、各类文献信息、翻译的外文资料、说明等,可作为论文的附录。“附录”二字格式要求同章标题。若有多个附录,则在章标题格式的附录内部,再分“附录一”“附录二”等,标题格式同节标题。

\subsection{致谢(或后记)}

致谢对象限于对完成论文有较重要学术帮助的团体和人士。谢辞谦虚诚恳,实事求是。“致谢”或“后记”两字中间空两个半角字符,格式要求同章标题用小二号黑体字,居中排版,内文格式同正文。

\subsection{页码}

页码一律设置在页面底端,居中,使用宋字五号阿拉伯数字。页码从绪论第一页起,一直延续至包括注释、参考书目和附录等在内的整篇论文结束,设置连续页码。

\section{论文(含绪论、正文、结论)篇幅}

\subsection{博士学位论文}

音乐学、作曲技术理论、音乐教育方向正文部分不少于10万字(不含谱例、图表、参考文献等);作曲、表演各方向正文部分不少于5万字(不含谱例、图表、参考文献等)。

\subsection{艺术学硕士学位论文}

音乐学、作曲技术理论、音乐教育、戏剧理论方向正文部分不少于4万字(不含谱例、图表、参考文献等);音乐学专业音乐文献翻译方向论文不少于2万字,译文10万字以上。作曲技术理论方向应与论文同时提交所论作品的乐谱与音响。音乐文献翻译方向的译文不作为正文篇幅计算字数,可用附录的形式附于正文后。

\subsection{艺术硕士专业学位论文}

艺术硕士研究生学位论文规范详见“艺教指委[2020]文件”《艺术硕士研究生专业学位论文写作规范》。

各表演方向(含指挥、声乐、管弦、民乐、现代器乐各研究方向)、乐器修造方向、作曲方向学位论文正文主体部分一般不少于0.8万字(不含谱例、图表);音乐教育学教学法各方向学位论文正文主体部分不得少于1.5万字(图表、谱例除外);艺术管理方向学位论文正文主体部分不得少于2.5万字(图表、谱例除外);电子科技方向(含电子音乐设计、录音艺术、音乐科技、数字媒体各研究方向)学位论文正文主体部分不得少于1.5 万字(不含谱例、图表),音乐科技学位论文应注重技术分析、实验数据验证,技术上可实现与可操作性的论文,提倡结合学位论文的技术性与艺术性展开研究。

\section{注释与参考文献的要求}

\subsection{注释}

注释是对论文中某一特定内容的补充说明,本规范要求对论文注释采用脚注即页下注形式。添加注释号时,使用word文档“插入”中的“引用”之“脚注”项,注文将自动列于页面底端,并与正文间用短横线加以分隔。

脚注在文内相应位置上用上标标注,页面底端加脚注,每页内连续编号,换页则重新标号。脚注编号用阿拉伯数字“1.2.3.”等,每页重新编号;脚注编号左顶格,文字用宋体、小五号,单倍行距(或固定值12磅);每条脚注单独成段,设悬挂缩进一个字符。

中文书名及刊名、文章名,皆用书名号。英文书名及刊名斜体,文章名加双引号。关于文献主要责任者,如果是多个责任者,中文姓名之间用顿号“、”分隔,英文姓名之间用半角的逗号“,”分隔。关于文献著作者的责任说明,一般的“著”“合著”应省略,特殊的“编”“主编”“译”等应在著作者姓名后注明。关于出版地,如果出版社名中包含了出版地,此项可省略。

\subsection{中文文献}

\subsubsection{著作}

1. 单本专著

作者名:《著作名》,出版社,×年第×版(第2版以上需标出版次,第1版则只需标出年),第×页(或第×-×页)。

示例:

于润洋:《现代西方音乐哲学导论》,湖南教育出版社,2000年,第56页(或第56—58页)。

2. 丛书分卷(册)

作者名:《丛书名》(第×卷或册),出版社,××××年第×版,第×页(或第×-×页)。

作者名:《书名》(《丛书名》第×卷或册),出版社,××××年第×版,第×页(或第×-×页)。

示例:

曹本冶、洛秦编著:《Ethnomusicology理论与方法英文文献导读》(卷一),上海音乐学院出版社,2019年,第35页。

3. 断代史及专门史中的章(节)

作者名:《书名》第×章×节《章或节名》,出版社,××××年第×版,第×页(或第×-×页)。

示例:

于润洋主编:《西方音乐通史》(中国艺术教育大系·音乐卷)第一章《从文艺复兴早期到若斯坎》,上海音乐出版社,2003年第3版,第22页。

4. 译著和古籍

外籍作者的国籍、我国清代及以前作者的朝代,需在作者名前注明,并用方括号[]括起。

[国籍或朝代]作者名:《著作名》,××(译者、点校者名)译(点校等),××校订,出版社,××××年,第×页(或×-×页)。

示例:

[美]保罗·亨利·朗:《西方文明中的音乐》,顾连理、张洪岛、杨燕迪、汤亚汀译,杨燕迪校订,贵州人民出版社,2001年,第35页。

[宋]陈旸:《乐书》,张国强点校,河南:中州古籍出版社,2019年,第1页。

\subsubsection{单篇论文}

1. 期刊论文

作者名:《论文题目》,载《期刊名》,××××年第×期。

示例:

黄翔鹏:《民间器乐曲实例分析与宫调定性》,载《中国音乐学》,1995年第3期。

2. 报纸文章

作者名:《论文题目》,载《报纸名》,××××年×月×日第×版。

示例:

史君良:《围绕旋律婉转歌唱》,载《音乐周报》,2002年11月21日第46期第3版。

3. 论文集中的单篇论文

作者名:《论文题目》,收录于×××编(或主编)《文集名》,出版社,××××年,第×页(或第×-×页)。

示例:

戴嘉枋:《中国近现代(当代)音乐史研究》,收录于杨燕迪主编《音乐学新论——音乐学的学科领域与研究规范》,北京:高等教育出版社,2011年,第35-55页。

4. 硕博学位论文

作者名:《论文题目》,××大学××方向硕(博)士学位论文,导师×××,××××年,第×页(或第×-×页)。

示例:

王赛:《夏野学术成果之研究》,上海音乐学院中国古代音乐史方向硕士学位论文,导8师洛秦教授,2003年,第10-26页。

\subsubsection{网络}

网络引用需尽量选择可信度高的官方网站,引用时需注明网站名称、网址和登陆时间。

示例:

中国非物质文化遗产网,http://www.ihchina.cn,登陆时间:2021-06-28。

\subsubsection{田野工作或采访}

田野工作或任何形式的访谈,需注明采访时间、地点、人物。

示例:

2021年6月6日,上海市汾阳路20号教学楼XX室,采访人:小音,被采访人:小乐。

\subsection{英文文献}

\textit{(注:其他外文文献,请尽可能参考英文文献格式处理。)}

注意英文文献中的书名和报刊名均为英文字母斜体,而不用书名号,英文中没有书名号。英文文献一律使用Times New Roman字体。

\subsubsection{著作}

1. 单本专著

作者,\textit{著作名} (出版社所在城市: 出版社, 年), 页码.

示例:

Tong Soon Lee, \textit{Chinese Street Opera in Singapore} (Urbana: University of Illinois Press, 2009), 9.

2. 报刊、音乐作品、唱片等正式出版物都需要斜体,例如:

Michel de Certeau, \textit{The Practice of Everyday Life} (Berkeley: University of California Press, 1984)

Suzanne Cusick, “Feminist Theory, \textit{Music Theory, and the Mind/Body Problem},” Perspectives of New Music 32/1 (1994): 8-27.

Robert Ito, “East Meets West, over Cocktails,” \textit{New York Times}, April 11, 2014.

Wen-Chung Zhou, \textit{Yü Ko} (C. F. Peters [P66098], 1965)

请注意:参考文献以作者姓(last name)的开头字母排序,则需要将姓放在前,以逗号分开,名排在后,例如:Lee, Tong Soon., \textit{Chinese Street Opera in Singapore} (Urbana: University of Illinois Press, 2009), 9.

\subsubsection{单篇论文}

1. 期刊论文

作者, “论文篇名,” \textit{刊名 }Vol/No (月年): 起止页码.

示例:

Suzanne Cusick, “Feminist Theory, Music Theory, and the Mind/Body Problem,” \textit{Perspectives of New Music} 32/1 (1994): 8-27.

或

Daphne Lei, “The Production and Consumption of Chinese Theatre in Nineteenth-Century California,” \textit{Theatre Research International} 28/3 (October 2003): 289-302.

2. 收录于文集中的单篇论文

作者, “文章,” in \textit{文集}, ed. 编者(出版信息), 起止页码.

示例:

Michael Broyles, “Immigrant, Folk and Regional Musics in the Nineteenth Century,” in \textit{The Cambridge History of American Music}, ed. David Nicholls (Cambridge: Cambridge University Press, 1998), 152.

\textbf{(务必注意:文集书名必须斜体,而且前面要有“in”,如上所示。)}

3. 博士论文

作者, “论文篇名,” 学位种类(如Ph.D.), 学位授予学校, ×年, 起止页码.

示例:

Theodore Dennis Brown, “A History and Analysis of Jazz Drumming to 1942,” Ph.D. diss., University of Michigan, 1976, 22-28.

\subsection{参考文献}

参考文献是作者写作论文时所参考和引用的文献资料。

参考文献置于文末。题头“参考文献”四个字位于页面上方,格式同章标题。参考文献应按类型划分,分为中文文献和外文文献两大类,注明标题“中文参考文献”和“外文参考文献”,标题格式同条名标题。

参考文献又可分为著作(含专著、编著、译注、论文集)、论文(含学位论文、报刊文章)、古籍等,用小四号字体的“(一)(二)(三)”加标题。

中文文献按作者姓名拼音字母顺序排序,序号采用阿拉伯数字后加下脚点的形式,即“1.2.3.”。同一作者两种或以上文献,以刊发时间先后排序。

西文参考文献按照作者姓氏字母(从A到Z)排序,即姓在前,名在后,其间以逗号分隔,名后面句号+逗号,英文及数字字体一律用Times New Roman。

示例:

Cusick, Kuzanne., “Feminist Theory, Music Theory, and the Mind/Body Problem,” \textit{Perspectives of New Music} 32/1 (1994)

Lee, Tong Soon., \textit{Chinese Street Opera in Singapore} (Urbana: University of Illinois Press, 2009)

Zhou, Wen-Chung., \textit{Yü Ko} (C. F. Peters [P66098], 1965)

序号及同一作者两种及以上文献与上述中文参考文献相同。

其他外文参考文献也同样按照作者姓名排序,请参照上述西文格式处理。

\section{谱例、插图与表格的格式}

\subsection{谱例}

论文引用的原版谱例要扫描,自己制作的谱例一般应用五线谱,建议使用打谱软件如Encore、Finale、Sibelius等进行制作。如有作者选择的特殊谱例,在记谱格式上要统一。论文中的谱例应按照章编号,如第一章第1个谱例,应标记为“谱例1-1”编号并设谱例小标题,谱例编号与标题用五号黑体字,置于谱例左上方,左缩进两个字符。

谱例上记载所引用的曲名及作者姓名放在右上角,如需标明演唱者、声部、乐器名、歌词以及其他说明文字,要齐全勿漏。单个谱例状态下的调号、节拍号、速度符号、力度符号等标记不要遗漏,应确保引用的乐谱符号书写完整而到位,凡是精简的谱例,应配以必要的注释予以说明。如果使用简谱记谱时,应书写调号、节拍号,且简谱音符上下表示音高的“点”的位置须准确。译著谱例上的术语,除保留意大利文速度、力度、表情术语外,其他都要译出。

\subsection{插图}

文中所用插图需清晰,插图的文本标示应包括两个主要部分:图题(含图序号、图题名)、图注。首先,图题,每幅用图应该施加专用标题,且所有插图应按照章编号,如第一章第1个插图,应标记为“图1-1”,图序号及图题名应在图的下方居中标出,使用小五号黑体字。其次,图注,所有插图均需有插图说明,可以行文或注释的形式说明图片的内容、提供者和来源,短小的图注也可以用夹注形式紧随图题之后排列或以变换字体的方式放置于图题之下。图题与图注中应包含图的责任者、拍摄时间及地点、图中内容阐释、图的出处等信息。如果所用的图为转引,而没有相应时间、地点和拍摄者信息,需要注明详尽的来源出处。插图必须紧跟文述。插图说明文字结束不用加句号。

\subsection{表格}

论文中出现的表格应用Excel或Word等制作的插入表格,不得手写或复印。表格应按照章编号,如第二章第1个表格,表号位“表2-1”,所有表格均需有表题(即表格标题),只有一个表格也需要编号。表格需居中。表号、表题置于表格上方一并居中,应加注释表明出处。表题用小五号、黑体字;表格内文字用小五号,楷体字并居中排列。表题和表中文字结尾不用加句号。


%\subsection{算法}
%
%算法环境可以使用 \pkg{algorithms} 或者 \pkg{algorithm2e} 宏包。
%
%\renewcommand{\algorithmicrequire}{\textbf{输入:}\unskip}
%\renewcommand{\algorithmicensure}{\textbf{输出:}\unskip}
%
%\begin{algorithm}[t]
%	\caption{Calculate $y = x^n$}
%	\label{alg1}
%	\small
%	\begin{algorithmic}
%		\REQUIRE $n \geq 0$
%		\ENSURE $y = x^n$
%		
%		\STATE $y \leftarrow 1$
%		\STATE $X \leftarrow x$
%		\STATE $N \leftarrow n$
%		
%		\WHILE{$N \neq 0$}
%		\IF{$N$ is even}
%		\STATE $X \leftarrow X \times X$
%		\STATE $N \leftarrow N / 2$
%		\ELSE[$N$ is odd]
%		\STATE $y \leftarrow y \times X$
%		\STATE $N \leftarrow N - 1$
%		\ENDIF
%		\ENDWHILE
%	\end{algorithmic}
%\end{algorithm}

\section{数字及标点符号用法}

\subsection{数字}

文中需要使用数字时,有阿拉伯数字和中文数字两种数字形式可以选用。

1. 通常情况下选用阿拉伯数字形式的情况有如下三种:用于计量的数字、编号的数字以及定型的含阿拉伯数字的词语。阿拉伯数字一律用 Times New Roman 字体。

示例:

2008年8月8日;总共256小节;第89小节;440Hz;MP3播放器。

2. 选用汉字数字的情况有如下三种:非公历纪年、概数、已定型的含汉字数字的词语。

示例:

秦文公四十四年;太平天国庚申十年九月二十四日;“五四”运动;七上八下;八九不离十

3. 两个数字连用表示概数时,两数之间不用顿号“、”隔开。

示例:

20世纪七八十年代;三四个月;五六十岁

4. 含有月日的专名采用汉字数字表示时,如果涉及一月、十一月、十二月,应用间隔号“\bullet”将表示月和日的数字隔开,涉及其他月份时,不用间隔号。

示例:

“一\bullet 二八”事变;“一二\bullet 九”运动;“五一”国际劳动节

5. 音乐专用名词与术语中的数字可做如下区分:专用名词用汉字,如九部乐、十二平均律;作品名包括乐曲名、书名一般一位数用汉字,二位数以上的用阿拉伯数字或汉字皆可,如《第九交响曲》《车尔尼钢琴练习曲50首》《1812年》序曲或《一八一二年》序曲;作品编号用阿拉伯数字,如作品5、作品101之2。

6. 引文标注中版次、卷次、页码等数字,除古籍应与所据版本一致外,一般均使用阿拉伯数字。

\subsubsection{标点符号}

除了标点符号常规用法外,有几点特别提醒如下:

1. 中英文标点符号使用可参照表\ref{tab:symbol}(中文正文的标点符号使用宋体的格式,英文标点符号使用Times New Roman格式,以下仅供参考):

\begin{table}
	\centering
	\caption{中英文标点符号格式}
	\begin{tabular}{ccc}
		\toprule
		标点符号 & 中文 & 英文 \\
		\midrule
		句号 & 。& . \\
		逗号 & ,& , \\
		问号 & ?& ? \\
		感叹号 & !&!\\
		省略号 & ……(居中)&...(在文字底部)\\
		破折号 & —— & —\\
		书名号 &《》 <> & 无 \\
		顿号 & 、& 无 \\
		冒号 & :& : \\
		分号 & ;& ; \\
		圆括号 &()& ( ) \\
		引号 & “”‘’ & “ ” ‘ ’\\
		连接号 & -、~、– & -\\
		\bottomrule
	\end{tabular}
	\label{tab:symbol}
\end{table}

2. 引号

双引号(“ ”)内部需要再使用引号时,用单引号(‘ ’)。书名号(《 》)内部需要再使用书名号时,用单括号(< >)。

3. 顿号

双引号与双引号之间,书名号与书名号之间连用时,中间无需加标点符号,如果插入了其他符号或文字则需要加上标点符号。

示例1:

“隐喻”“展演”“范式”;《中国古代音乐史简编》《西方音乐史》

示例2:

“隐喻”①、“展演”②、“范式”③;夏野的《中国古代音乐史》、格劳特的《西方音乐史》

4. 注释号

如果注释内容是引文的出处或是整个引文内容相关,注释号通常放在后引号之后;如果注释内容与引号内部某个词语或局部内容相关通常放在引号内所注释词语后。

5. 连接号

需要区别三种连接号,短横线“-”、一字线“—”和波纹线“~”。

表示编号或者复合名词时通常使用短横线“-”。如:表2-8;吐鲁番-哈密盆地;夏尔·卡米尔·圣-桑。

标示相关项目(如时间、地域等)的起止,标示数值范围的起止,通常用一字线“—”,也可以用波纹线“~”,但全文同类情况应注意统一。如:沈括(1031—1095),宋朝人;第5~8小节。

英文书写时也应区分使用连接号,英文参考文献页码起止的连线书写应为短横线“-”,如:David Lewin, “Klumpenhouwer Networks and Some Isographies that Involve Them, ” \textit{Music Theory Spectrum} 12/1 (1900), 83-120.

英文表达地名之间或时间之间的“至”意时,可用短横线“-”、一字线“—”,但不可用中文破折号(二字格)“——”,如:the 1914 -18 war; the Hong Kong—Kowloon ferry.

\section{音乐术语的规范}

\subsection{作品编号}

音乐作品的作品编号可使用以拉丁文缩写的Op.或Opp.(复数形式)加上顺序编码的阿拉伯数字,如Op. 55。如果使用特定音乐学家或者出版商名姓来做出作品编号的话,应该以首字母的正确写法来书写,后面可不再缀用标点符号,如K551、SWV3。

\subsection{力度术语和符号}

段落强弱的标记,往往使用缩写的字母形式,如 \textit{p}(\textit{piano},弱)、\textit{pp}(\textit{pianissimo},很弱)、\textit{f}(\textit{forte},强)、\textit{ff}(\textit{fortissimo},很强)等;

渐变强弱与突变强弱的标记,可使用意大利文或者其缩写,如 \textit{cresc.}(\textit{crescendo},渐强)、\textit{dim.}(\textit{diminuendo},渐弱)、\textit{fp}(\textit{forte - piano},强后即弱)、\textit{sf}(\textit{sforzando},特强)等;同时,也可以使用力度符号做标记,如>(重音记号)。

需注意的是,不管是在论文的行文中还是在乐谱中出现的力度术语和缩略语,都需要以小写字母书写并以斜体形式呈现,以区别于其他的字符内容。

\subsection{速度术语和表情术语}

用于表示整首作品或某一段落作品的固定的速度术语和符号,书写时要求放置在整部作品(或段落)起始之处的节拍号上方,使用正体加黑的字体记写,且首字母还要采用大写形式,如 Largo(广板)、Andante(行板)、Allegro(快板)等。

用于表示音乐作品中临时或过渡性的速度变化的术语,要求使用小写字母并且以斜体的方式记写,术语的首字母必须处于对准速度开始变化的音符的上方,如 rall.(减慢渐弱)、a tempo(恢复原速)。

表情术语一般使用意大利文标记,如果单独使用表情术语,首字母要大写,如果是记写在速度记号之后的表情术语,则要用小写形式(例如 Andante cantabile,如歌的行板),在乐谱起始处使用时,要用正体字符书写,乐谱中间使用时则以小写并斜体的方式。

\subsection{乐谱中的其他术语和记号}

除了一般的乐谱记写格式外,仍有需要注意的两个方面:一是反复记号,首字母以大写形式,词或字母之后应后缀以下脚点,如 D.C.(从头开始)、D.S.(从记号出反复);二是八度记号,字母均要以小写形式书写,词后不缀写下脚点,如 8va(或写作“8”)、con8等。

\section{论文打印及装订要求}

\subsection{正文}

学位论文的文字内容可双面印刷。

\subsection{封面}

博士论文封面、封底采用浅灰 ( 冷色调 ) 色A3或A4型号120克胶版纸。硕士论文封面、封底采用浅黄色A3或A4型号120克胶版纸。其他文本采用白色A4型号80克打印纸。

\subsection{书脊}

论文装订后应有书脊,书脊处按照封面模版的提示要求进行填写。\textit{(注:论文篇幅达不到制作书脊者,不在此列。)}
