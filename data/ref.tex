% !TeX root = ../shcmthesis-example.tex

\chapter*{参考文献}
\addcontentsline{toc}{chapter}{参考文献}

% 参考文献是作者写作论文时所参考和引用的文献资料。
% 参考文献置于文末。题头“参考文献”四个字位于页面上方,格式同章标题。
% 参考文献应按类型划分,分为中文文献和外文文献两大类,注明标题“中文参考文献”和“外文参考文献”,标题格式同条名标题。
% 参考文献又可分为著作(含专著、编著、译注、论文集)、论文(含学位论文、报刊文章)、古籍等,用小四号字体的“(一)(二)(三)”加标题。

\subsection*{一、中文参考文献}
% 中文文献按作者姓名拼音字母顺序排序,序号采用阿拉伯数字后加下脚点的形式,即“1.2.3.”。
% 同一作者两种或以上文献,以刊发时间先后排序。

\subsubsection*{(一)著作}
% 1. 单本专著
% 作者名:《著作名》,出版社,×年第×版(第2版以上需标出版次,第1版则只需标出年),第×页(或第×-×页)。
% 2. 丛书分卷(册)
% 作者名:《丛书名》(第×卷或册),出版社,××××年第×版,第×页(或第×-×页)。
% 作者名:《书名》(《丛书名》第×卷或册),出版社,××××年第×版,第×页(或第×-×页)。
% 3. 断代史及专门史中的章(节)
% 作者名:《书名》第×章×节《章或节名》,出版社,××××年第×版,第×页(或第×-×页)。
% 4. 译著和古籍
% 外籍作者的国籍、我国清代及以前作者的朝代,需在作者名前注明,并用方括号[]括起。
% [国籍或朝代]作者名:《著作名》,××(译者、点校者名)译(点校等),××校订,出版社,××××年,第×页(或×-×页)。

\begin{achievements}
\item 于润洋:《现代西方音乐哲学导论》,湖南教育出版社,2000年,第56页(或第56—58页)。
\item 曹本冶、洛秦编著:《Ethnomusicology理论与方法英文文献导读》(卷一),上海音乐学院出版社,2019年,第35页。
\item 于润洋主编:《西方音乐通史》(中国艺术教育大系·音乐卷)第一章《从文艺复兴早期到若斯坎》,上海音乐出版社,2003年第3版,第22页。
\end{achievements}

\subsubsection*{(二)单篇论文}
% 1. 期刊论文
% 作者名:《论文题目》,载《期刊名》,××××年第×期。
% 2. 报纸文章
% 作者名:《论文题目》,载《报纸名》,××××年×月×日第×版。
% 3. 论文集中的单篇论文
% 作者名:《论文题目》,收录于×××编(或主编)《文集名》,出版社,××××年,第×页(或第×-×页)。
% 4. 硕博学位论文
% 作者名:《论文题目》,××大学××方向硕(博)士学位论文,导师×××,××××年,第×页(或第×-×页)。

\begin{achievements}
\item 黄翔鹏:《民间器乐曲实例分析与宫调定性》,载《中国音乐学》,1995年第3期。
\item 史君良:《围绕旋律婉转歌唱》,载《音乐周报》,2002年11月21日第46期第3版。
\item 戴嘉枋:《中国近现代(当代)音乐史研究》,收录于杨燕迪主编《音乐学新论——音乐学的学科领域与研究规范》,北京:高等教育出版社,2011年,第35-55页。
\item 王赛:《夏野学术成果之研究》,上海音乐学院中国古代音乐史方向硕士学位论文,导师洛秦教授,2003年,第10-26页。
\end{achievements}

\subsubsection*{(三)网络}
% 网络引用需尽量选择可信度高的官方网站,引用时需注明网站名称、网址和登陆时间

\begin{achievements}
\item 中国非物质文化遗产网,http://www.ihchina.cn,登陆时间:2021-06-28。
\end{achievements}

\subsubsection*{(四)田野工作或采访}
% 田野工作或任何形式的访谈,需注明采访时间、地点、人物。

\begin{achievements}
	\item 2021年6月6日,上海市汾阳路20号教学楼XX室,采访人:小音,被采访人:小乐。
\end{achievements}


\subsection*{二、英文参考文献}
% 西文参考文献按照作者姓氏字母(从A到Z)排序,即姓在前,名在后,其间以逗号分隔,名后面句号+逗号,英文及数字字体一律用Times New Roman。
% 序号及同一作者两种及以上文献与上述中文参考文献相同。
% 其他外文参考文献也同样按照作者姓名排序,请参照上述西文格式处理。

\subsubsection*{(一)著作}
% 1. 单本专著
% 作者,\textit{著作名} (出版社所在城市: 出版社, 年), 页码.
% 2. 报刊、音乐作品、唱片等正式出版物都需要\textit{斜体},
% 
% 请注意:参考文献以作者姓(last name)的开头字母排序,则需要将姓放在前,以逗号分开,名排在后

\begin{achievements}
\item Tong Soon Lee, \textit{Chinese Street Opera in Singapore} (Urbana: University of Illinois Press, 2009), 9.
\item Michel de Certeau, \textit{The Practice of Everyday Life} (Berkeley: University of California Press, 1984)
\item Suzanne Cusick, “Feminist Theory, Music Theory, and the Mind/Body Problem,” \textit{Perspectives of New Music} 32/1 (1994): 8-27.
\item Robert Ito, “East Meets West, over Cocktails,” \textit{New York Times}, April 11, 2014.
\item Wen-Chung Zhou, \textit{Yü Ko} (C. F. Peters [P66098], 1965)
\end{achievements}

\subsubsection*{(二)单篇论文}
% 1. 期刊论文
% 作者, “论文篇名,” \textit{刊名} Vol/No (月年): 起止页码.
% 2. 收录于文集中的单篇论文
% 作者, “文章,” in \textit{文集}, ed. 编者(出版信息), 起止页码.
% (务必注意:文集书名必须斜体,而且前面要有“in”,如上所示。)
% 3. 博士论文
% 作者, “论文篇名,” 学位种类(如Ph.D.), 学位授予学校, ×年, 起止页码.

\begin{achievements}
\item Suzanne Cusick, “Feminist Theory, Music Theory, and the Mind/Body Problem,” \textit{Perspectives of New Music} 32/1 (1994): 8-27.
\item Daphne Lei, “The Production and Consumption of Chinese Theatre in Nineteenth-Century California,” \textit{Theatre Research International }28/3 (October 2003): 289-302.
\item Michael Broyles, “Immigrant, Folk and Regional Musics in the Nineteenth Century,” in \textit{The Cambridge History of American Music}, ed. David Nicholls (Cambridge: Cambridge University Press, 1998), 152.

\end{achievements}