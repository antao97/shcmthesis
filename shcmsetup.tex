% !TeX root = ./shcmthesis-example.tex

% 论文基本信息配置

\thusetup{
  %******************************
  % 注意:
  %   1. 配置里面不要出现空行
  %   2. 不需要的配置信息可以删除
  %   3. 建议先阅读文档中所有关于选项的说明
  %******************************
  %
  % 输出格式
  %   选择打印版(print)或用于提交的电子版(electronic),前者会插入空白页以便直接双面打印
  %
  output = electronic,
  %
  %
  % 封面格式
  %	  选择用于预审(preliminary)或用于最终提交的终版(final)
  %
  cover = final,
  %
  %
  % 学位类型
  %   选择本科(bachelor)、硕士(master)或者博士(doctor)
  %
  degree = doctor,
  %
  %
  % 论文标题
  %   -1和-2分别对应标题的第一行和第二行,加*的为英文标题
  %
  title-1  = {上海音乐学院学位论文},
  title-2  = {\LaTeX{} 模板 v\version},
  title-1* = {\LaTeX{} v\version~Thesis Template},
  title-2* = {Shanghai Conservatory of Music},
  %  
  %  
  % 论文编号
  %
  thesis-id = {001},
  % 
  % 
  % 学校代码
  % 
  school-id = {10278},
  %
  %
  % 学科专业
  %
  discipline  = {音乐与舞蹈学/中国音乐研究},
  %
  % 研究方向
  %
  professional-field  = {中国传统音乐理论},
  %
  % 作者姓名
  %
  author  = {王丹丹},
  %
  % 作者学号
  %
  student-id = {D202000},
  %
  % 指导教师
  %   中文姓名和职称之间以\quad分开
  %
  supervisor  = {李红\quad 教授},
  %
  % 完成日期
  %
  date = {2025年6月},
  %
  % 是否在中文封面后的空白页生成书脊(默认 false)
  %
  include-spine = false,
  %
}

% 载入所需的宏包

% 定理类环境宏包
\usepackage{amsthm}
% 也可以使用 ntheorem
% \usepackage[amsmath,thmmarks,hyperref]{ntheorem}

\thusetup{
  %
  % 数学字体
  % math-style = GB,  % GB | ISO | TeX
  math-font  = xits,  % stix | xits | libertinus
}

% 可以使用 nomencl 生成符号和缩略语说明
% \usepackage{nomencl}
% \makenomenclature

% 表格加脚注
\usepackage{threeparttable}

% 表格中支持跨行
\usepackage{multirow}

% 固定宽度的表格。
% \usepackage{tabularx}

% 跨页表格
\usepackage{longtable}

% 算法
\usepackage{algorithm}
\usepackage{algorithmic}

% 代码块
\usepackage{pythonhighlight}

% 量和单位
\usepackage{siunitx}

% 参考文献使用 BibTeX + natbib 宏包
% 顺序编码制
% \usepackage[sort]{natbib}

% 参考文献使用 BibLaTeX 宏包
% \usepackage[style=gb7714-2015]{biblatex}
% 声明 BibLaTeX 的数据库
% \addbibresource{ref/refs.bib}

% 定义所有的图片文件在 figures 子目录下
\graphicspath{{figures/}}

% 数学命令
\makeatletter
\newcommand\dif{%  % 微分符号
  \mathop{}\!%
  \ifthu@math@style@TeX
    d%
  \else
    \mathrm{d}%
  \fi
}
\makeatother

% hyperref 宏包在最后调用
\usepackage{hyperref}
